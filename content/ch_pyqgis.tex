\chapter{PyQGIS}

\pagestyle{fancy}
\fancyhf{}
\fancyhead[OC]{\leftmark}
\fancyhead[EC]{\rightmark}
%\renewcommand{\footrulewidth}{1pt}
\cfoot{\thepage}

\section{resources}

Resources for next steps.

Programming this in python (PyQGIS)

\url{opensourceoptions.com}
\url{https://opensourceoptions.com/blog/loading-and-symbolizing-vector-layers/}

OPEN PYTHON CONSOLE

Open the python console using the icon in the toolbar
\begin{tabular}{@{}c@{}}\includegraphics[width=4ex]{images/python_console_icon.png}\end{tabular}

ADD VECTOR TO QGIS

In browser panel, navigate to the shapefile, right click $\rightarrow$ Layer properties.

Copy the path and assign it to a variable.

location = $r"/home/kerry/Documents/QGIS/Counties_and_Unitary_Authorities_December_2016_Full_Clipped_Boundaries_in_England_and_Wales/Counties_and_Unitary_Authorities_December_2016_Full_Clipped_Boundaries_in_England_and_Wales.shp"$

iface.addVectorLayer(location, "", "ogr")

layer = iface.activeLayer()
layer.name()

CHANGE SYMBOLOGY
# create a new symbol
$symbol = QgsLineSymbol.createSimple({'line_style': 'dash', 'color': 'red'})$


For lines use: QgsLineSymbol
For points use: QgsMarkerSymbol
For polygons use: QgsFillSymbol

# apply symbol to layer renderer
layer.renderer().setSymbol(symbol)

# repaint the layer
layer.triggerRepaint()


Find the options have to play with.
print(layer.renderer().symbol().symbolLayers()[0].properties())

#output for line
{'capstyle': 'square', 'customdash': '5;2', 'customdash_map_unit_scale': '3x:0,0,0,0,0,0', 'customdash_unit': 'MM', 'draw_inside_polygon': '0', 'joinstyle': 'bevel', 'line_color': '152,125,183,255', 'line_style': 'solid', 'line_width': '0.26', 'line_width_unit': 'MM', 'offset': '0', 'offset_map_unit_scale': '3x:0,0,0,0,0,0', 'offset_unit': 'MM', 'use_custom_dash': '0', 'width_map_unit_scale': '3x:0,0,0,0,0,0'}

#output for polygon
{'border_width_map_unit_scale': '3x:0,0,0,0,0,0', 'color': '243,166,178,255', 'joinstyle': 'bevel', 'offset': '0,0', 'offset_map_unit_scale': '3x:0,0,0,0,0,0', 'offset_unit': 'MM', 'outline_color': '35,35,35,255', 'outline_style': 'solid', 'outline_width': '0.26', 'outline_width_unit': 'MM', 'style': 'solid'}

symbol = QgsFillSymbol.createSimple({'outline_style': 'dash', 'color': 'red'})
layer.renderer().setSymbol(symbol)
layer.triggerRepaint()


layer = QgsVectorLayer(location, 'name', 'ogr')

for field in layer.fields():
	print(field.name(), field.type())
	
	
	
	
	

location = r"/home/kerry/Documents/QGIS/Counties_and_Unitary_Authorities_December_2016_Full_Clipped_Boundaries_in_England_and_Wales/Counties_and_Unitary_Authorities_December_2016_Full_Clipped_Boundaries_in_England_and_Wales.shp"

%iface.addVectorLayer(location, "", "ogr")
layer = QgsVectorLayer(location, 'name', 'ogr')
layer.name()

for field in lyr.fields():
	print(field.name(), field.type())


CHANGE SYMBOLOGY
# create a new symbol
symbol = QgsLineSymbol.createSimple({'line_style': 'dash', 'color': 'red'})


For lines use: QgsLineSymbol
For points use: QgsMarkerSymbol
For polygons use: QgsFillSymbol

# apply symbol to layer renderer
layer.renderer().setSymbol(symbol)

# repaint the layer
layer.triggerRepaint()


Find the options have to play with.
print(layer.renderer().symbol().symbolLayers()[0].properties())

#output for line
{'capstyle': 'square', 'customdash': '5;2', 'customdash_map_unit_scale': '3x:0,0,0,0,0,0', 'customdash_unit': 'MM', 'draw_inside_polygon': '0', 'joinstyle': 'bevel', 'line_color': '152,125,183,255', 'line_style': 'solid', 'line_width': '0.26', 'line_width_unit': 'MM', 'offset': '0', 'offset_map_unit_scale': '3x:0,0,0,0,0,0', 'offset_unit': 'MM', 'use_custom_dash': '0', 'width_map_unit_scale': '3x:0,0,0,0,0,0'}

#output for polygon
{'border_width_map_unit_scale': '3x:0,0,0,0,0,0', 'color': '243,166,178,255', 'joinstyle': 'bevel', 'offset': '0,0', 'offset_map_unit_scale': '3x:0,0,0,0,0,0', 'offset_unit': 'MM', 'outline_color': '35,35,35,255', 'outline_style': 'solid', 'outline_width': '0.26', 'outline_width_unit': 'MM', 'style': 'solid'}

symbol = QgsFillSymbol.createSimple({'outline_style': 'dash', 'color': 'red'})
layer.renderer().setSymbol(symbol)
layer.triggerRepaint()




~LOAD IN CSV FILE

from qgis.PyQt import QtGui

location = r"/home/kerry/Documents/QGIS/Counties_and_Unitary_Authorities_December_2016_Full_Clipped_Boundaries_in_England_and_Wales/Counties_and_Unitary_Authorities_December_2016_Full_Clipped_Boundaries_in_England_and_Wales.shp"

if 1 == 2:
iface.addVectorLayer(location, "", "ogr")

layer = iface.activeLayer()
layer.name()

#CHANGE SYMBOLOGY
symbol = QgsFillSymbol.createSimple({'outline_style': 'dash', 'color': 'red'})
layer.renderer().setSymbol(symbol)
layer.triggerRepaint()
else:
layer = QgsVectorLayer(location, 'name', 'ogr')

for field in layer.fields():
print(field.name(), field.type())

targetField = 'objectid'
rangeList = []
opacity = 1

# define value ranges
minVal = 0.0
maxVal = 5.0

# range label
lab = 'Group 1'

# color (yellow)
rangeColor = QtGui.QColor('#ffee00')

# create symbol and set properties
symbol1 = QgsSymbol.defaultSymbol(layer.geometryType())
symbol1.setColor(rangeColor)
symbol1.setOpacity(opacity)

#create range and append to rangeList
range1 = QgsRendererRange(minVal, maxVal, symbol1, lab)
rangeList.append(range1)



# define value ranges
minVal = 5.1
maxVal = 80

# range label
lab = 'Group 2'

# color (yellow)
rangeColor = QtGui.QColor('#00eeff')

# create symbol and set properties
symbol2 = QgsSymbol.defaultSymbol(layer.geometryType())
symbol2.setColor(rangeColor)
symbol2.setOpacity(opacity)

#create range and append to rangeList
range2 = QgsRendererRange(minVal, maxVal, symbol2, lab)
rangeList.append(range2)

# create the renderer
groupRenderer = QgsGraduatedSymbolRenderer('', rangeList)
groupRenderer.setMode(QgsGraduatedSymbolRenderer.EqualInterval)
groupRenderer.setClassAttribute(targetField)

# apply renderer to layer
layer.setRenderer(groupRenderer)

# add to QGIS interface
QgsProject.instance().addMapLayer(layer)




#add delimited text alyer
#https://www.geodose.com/2018/07/python-qgis-tutorial-adding-csv-data.html

csvfile = "/home/kerry/GitLab/police_geography/training_datasets/sw_5forces_street_by_lsoa.csv"
#?encoding=%s&delimiter=%s&xField=%s&yField=%s&crs=%s" % ("UTF-8",",", "Longitude", "Latitude","epsg:4326")

csv_layer=QgsVectorLayer(csvfile, "street", "delimitedtext")

#Check if layer is valid
if not csv_layer.isValid():
print ("Layer not loaded")

#Add CSV data    
QgsProject.instance().addMapLayer(csv_layer)




