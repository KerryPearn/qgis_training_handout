\chapter{Summary}

\pagestyle{fancy}
\fancyhf{}
\fancyhead[OC]{\leftmark}
\fancyhead[EC]{\rightmark}
%\renewcommand{\footrulewidth}{1pt}
\cfoot{\thepage}

Here are the main take away points for this QGIS tutorial.\\

\begin{enumerate}
	\item 
	
Add data (layers) to your QGIS project.

For your own data from a csv file, add as a delimited file 
\begin{tabular}{@{}c@{}}\includegraphics[width=4ex]{images/add_delimited_text_layer_icon.png}\end{tabular}

\begin{enumerate}
	\item 
For point data:
If your data contains Latitude \& Longitude, or Easting \& Northing, use these fields to plot your point data.

	\item 

For polygon data:
If your data contains a unique reference code for a spatial area, use a corresponding shapefile (do a google search to find the shapefile you need, and download it). Add the shapefile to your QGIS project as a vector layer \begin{tabular}{@{}c@{}}\includegraphics[width=4ex]{images/add_vector_layer_icon.png}\end{tabular}

Join your csv data to the shapefile: double click on the layer name in the \textit{Layers panel} $\rightarrow$ Select \textbf{Join} in the LHS pane.\\
Initiate a new join \begin{tabular}{@{}c@{}}\includegraphics[width=4ex]{images/green_button_icon.png}\end{tabular}

\end{enumerate}

	\item 
For each data layer, each feature on the map canvas (point or polygon) has a row of data in the attribute table \begin{tabular}{@{}c@{}}\includegraphics[width=4ex]{images/attribute_table_icon.png}\end{tabular}. 
	\item 
Become familiar with your data layers (both within the map canvas and attribute table). Calculate any new fields in your attribute table that you want to visualise (or choose to do these within the relevant expression dialogs along the way \begin{tabular}{@{}c@{}}\includegraphics[width=4ex]{images/expression_icon.png}\end{tabular}
).

	\item 
Add symbology to your layers (their visual appearance) using the Layer Styling panel \begin{tabular}{@{}c@{}}\includegraphics[width=4ex]{images/layer_styling_panel_icon.png}\end{tabular}.
\begin{enumerate}
	\item
	Single symbol (uniform appearance)
	\item
	Categorized (for categorical fields)
	\item
	Graduated (for continuous numerical fields)
	\item
	Heatmap or Point cluster (for dense point data).
\end{enumerate}
	\item 
Include a basemap to give your data context
\begin{enumerate}
	\item 
Use a shapefile and give it appropriate \textit{single symbol} symbology (either use your own shapefile, or in the coordinate field type \textit{world})
	\item 
Or in Browser Panel select \textit{XYZ Tiles} $\rightarrow$ \textit{OpenStreetMap}
\end{enumerate}
	\item 
Export your map.
\begin{enumerate}
	\item 
Snipping/shutter
	\item 
Menu: Project $\rightarrow$ Import/Export
	\item 
Menu: Project $\rightarrow$ New Print Layout.

\end{enumerate}

\end{enumerate}
