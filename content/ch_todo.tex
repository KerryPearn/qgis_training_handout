\chapter{To Do}

\pagestyle{fancy}
\fancyhf{}
\fancyhead[OC]{\leftmark}
\fancyhead[EC]{\rightmark}
%\renewcommand{\footrulewidth}{1pt}
\cfoot{\thepage}

%%%%%%%%%%%%%%%%%%%%%%%%%%%%%%%%%%%%%%%%%%%%%%%%%%%%%%%%%%%
%%%%%%%%%%%%%%%%%%%%%%%%%%%%%%%%%%%%%%%%%%%%%%%%%%%%%%%%%%%
Feedback from the 25/02/2020 session

1. Projections need to covered more. E&N point data - which projection to use?
2. QGIS automatically does reprojection on the fly.


Automating dull stuff with Graphical Modeler
Just to be reminded of what it takes to make a great model in every sense of the word: input, algorithms and output. And when you can string these processes together in graphical modeler, suddenly life gets easier because you can reuse them in the future.
QGIS 3 refinements for graphical modeler solidifies it as a key choice for automating. For example, this simple model sets an input parameter, buffers a set of points and generates an output.
From start to finish, it runs quick and smoothly without any snags along the way. And instead of a shapefile for output, it generates the new default geopackage output.

view dbf in ubuntu. Tried dbview...maybe need to restart comp
\url{}https://askubuntu.com/questions/687575/dbf-viewer-for-ubuntu}

And projections - the non-uniform globe, to flat

Show histograms first for style ranges.

Discuss the input file. That all columns are about similar data: count of crimes. So if want to plot more than one of these columns to take extra time to consider a colour scale that would be suitable for them all. For this case, let's take both Total Crime, and ASB. Make them do a colour style for full range first. Not twice. But duplicate layer to choose a different column. Think about pros and againsts of 1 common colour vs unique for each column.

Print Layout: lock layer (stops it being redrawn)

go straight to layer styling




